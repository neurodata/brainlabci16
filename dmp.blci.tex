
Our data management plan covers several distinct classes of data: 
\begin{itemize}
\addtolength{\itemsep}{-5pt}
\item neuroscience data contributed by the open science community,
\item software and documentation developed by the project team, and 
\item operational data, such as logging and usage statistics.
\end{itemize}
\vspace{-5pt}
All data sets will be archival; they are intended for long-term preservation and
subject to the data management policies described below.

This project is primarily conducted out of the KAVLI Neuroscience Discovery 
Institute.  Because it is a new institute, we are in the process of codifying 
data management principles.  We adopt the data management practices of 
the Institute for Data-Intensive Science and Engineering (IDIES) of which 
Burns, Vogelstein, and Miller are all active members.  IDIES aggregates the
data management and scientific computing exerptise at Johns Hopkins.

IDIES has successfully deployed data-intensive Web services that contribute 
computation and analysis to the public for several disciplines, including astronomy (sdss.org), 
computational fluids (turbulence.pha.jhu.edu), and connectomics 
(openconnecto.me). IDIES hosts more than 11 petabytes of contributed data in support of 
many disciplines.  IDIES has 13 years of experience 
hosting publicly accessible scientific databases and data-intensive Web-services. The best 
practices of the Institute will be adopted, as described below.

\subsubsection*{Licenses}
All scientific data sets will be released under the {\em Open Data Commons Attribution} (ODC-By) license.
The license allows for free use of all data products, including copying, sharing, and redistributing data, and
deriving new, customized data products.  Our specific license terms require that all subsequent
data use attributes the original publication that contributed the data as well as the project 
that hosts the data.  
All software and course material will be released under the Apache 2.0 license. This license is 
permissive; it places few restrictions on the use of software.
Derived software products or software systems that integrate our code or course materials 
do not need to be open-source.

\para{Documentation and Metadata} Source code documentation will be collocated and released with 
software, governed by the same open source license, and hosted on public repositories. 
Documentation for datasets are treated similarly. They 
will be managed in version control public repositories. They will be available as links that are 
collocated with the Web-services that provide access to data. These are already standard practices 
of the collaborators. 

\subsubsection*{Logging and Reporting Data}
We will use IDIES' extensive logging infrastructure to  track scientists
use data-intensive Web-services and supercomputing allocations in their research.
One of the most useful byproducts of the Sloan Digital Sky Survey  data has been the 
usage logs that we have kept since
the very beginning of the project: every Web hit and every single query have been logged since Day 1.
Today, the log database is over 2TB, and contains rich historical information about how astronomers
learned to access a virtual telescope. 
This has resulted in an amazingly rich and useful resource
for SDSS scientists and project managers and, because the dataset is available to anyone, many other
projects and researchers. IDIES has generalized this logging infrastructure for all projects.

Based on our logging infrastructure, we will be able to analyze the data usage patterns 
and contributions.  The goal is to inform the evaluation and reporting process 
with data-driven metrics at an arbitrarily fine granularity.  

\para{Data Integrity} IDIES systems make several efforts to ensure the integrity of the
 data and to prevent malicious or unintended modification. We use RESTful interfaces 
 for data access and to execute analysis routines, ensuring that all interfaces are functional and 
 do not modify the original data. For data stored at IDIES, we also regularly crawl our data 
 checking content against pre-computed checksums to protect against corruption from hardware and software errors. 
 For data stored in the Amazon cloud, we will rely on the integrity of AWS's data services. 
 

\para{Security, Privacy, and Embargo}
  The privacy of data sets is implemented with a combination of a user authentication in 
  our project management system and the secure Web-services (\url{https://}).  
  The combination allows users to self-manage access control to their data.  They may 
  set individual data sets, images, and annotation projects as public or private.
  Derived data products may be kept private by registered users of the system. 
  Private data sets will be encrypted end-to-end for all Web-services and visualization tools.

  The contributors of data maintain control of the data even after placing it in the system.
  They can determine when to make data public, when to release data from embargo, etc.

\para{Storage and Backup During the Project} The project builds a prototype that stores 
data in the Amazon Cloud.  Amazon provides guarantees about the reliability of data
in the service level agreements, e.g.~some data services implement triply redundant storage. 

\para{Long-Term Archival and Preservation}  While the goal of this project is to deploy a 
prototype, we will accumulate valuable data sets in doing so.  At the end of the project,
we will ingest these data sets in to the IDIES' storage infrastructure,
IDIES has preserved all scientific data that has 
  been ingested and will continue to do so in perpetuity. 
  The first resources developed were the Sloan Digital Sky Survey (sdss.org). 
  Multiple strategies have been used to maintain these data. We 
  have partnered with Google to store all image data on an ongoing basis. The development of 
  new data-intensive clusters (GrayWulf 2006 and Data-Scope 2012) has provisioned a small 
  fraction of resources to maintain the entirety of our previously collected data. Our preservation 
  ethic includes preserving the function and semantics of the data long after project operation 
  completes. We will do so by defining archival packages that include algorithms and methods as 
  well as data. We recognize sustainable preservation as one of the most challenging aspects of 
  developing data resources and note our commitment and experience in defining strategies to 
  fund and maintain resource-sharing beyond project lifetimes. 

\para{Data Sharing and Dissemination} This project focuses on making public datasets broadly 
available, leveraging the team’s expertise in storage systems and data 
processing to host scientific datasets for the nueroscience community.
This process makes data a community resource, greatly enhancing their 
utility. All public datasets referenced in this project will be available through Web-services. 

\para{Ownership, Copyright, Intellectual Property} Our projects encourages scientists and communities 
that ``open source'' their data in exchange for storage and analysis services. Open-source data 
becomes a community resource, unrestricted for non-commercial use. Data providers retain 
copyright privileges and reserve licensing and approval rights for commercial uses of data. 
Users of the data reserve rights to their algorithms and analysis techniques.

